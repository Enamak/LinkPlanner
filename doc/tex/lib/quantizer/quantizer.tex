\clearpage

\section{Quantizer}

\begin{tcolorbox}	
	\begin{tabular}{p{2.75cm} p{0.2cm} p{10.5cm}} 	
		\textbf{Header File}   &:& quantizer$\_*$.h \\
		\textbf{Source File}   &:& quantizer$\_*$.cpp \\
        \textbf{Version}       &:& 20180423 (Celestino Martins) \\
	\end{tabular}
\end{tcolorbox}

This block simulates a quantizer, where the signal is quantized into discrete levels. Given a quantization bit precision, $resolution$, the outputs signal will be comprise $2^{nBits}-1$ levels.

\subsection*{Input Parameters}

\begin{table}[h]
	\centering
	\begin{tabular}{|c|c|c|c|c|cccc}
		\cline{1-4}
		\textbf{Parameter} & \textbf{Unity} & \textbf{Type} & \textbf{Values} &   \textbf{Default}& \\ \cline{1-5}
		resolution & bits  & double & any & $inf$ \\ \cline{1-5}	
        maxValue   & volts & double & any & $1.5$ \\ \cline{1-5}	
        minValue   & volts & double & any    & $-1.5$ \\ \cline{1-5}	
	\end{tabular}
	\caption{Quantizer input parameters}
	\label{table:quantizer_in_par}
\end{table}


\subsection*{Methods}

Quantizer() {};
\bigbreak
Quantizer(vector$<$Signal *$>$ \&InputSig, vector$<$Signal *$>$ \&OutputSig) :Block(InputSig, OutputSig)\{\};
\bigbreak
void initialize(void);
\bigbreak
bool runBlock(void);
\bigbreak
void setSamplingPeriod(double sPeriod) { samplingPeriod = sPeriod; }
\bigbreak
void setSymbolPeriod(double sPeriod) { symbolPeriod = sPeriod; }
\bigbreak
void setResolution(double nbits) { resolution = nbits; }
\bigbreak
double getResolution() { return resolution; }
\bigbreak
void setMinValue(double maxvalue) { maxValue = maxvalue; }
\bigbreak
double getMinValue() { return maxValue; }
\bigbreak
void setMaxValue(double maxvalue) { maxValue = maxvalue; }
\bigbreak
double getMaxValue() { return maxValue; }

\subsection*{Functional description}

This block can performs the signal quantization according to the defined input parameter \textit{resolution}.

Firstly, the parameter \textit{resolution} is checked and if it is equal to the infinity, the output signal correspond to the input signal. Otherwise, the quantization process is applied. The input signal is quantized into $2^{resolution-1}$ discrete levels using the standard \textit{round} function.

\pagebreak
\subsection*{Input Signals}

\subparagraph*{Number:} 1

\subsection*{Output Signals}

\subparagraph*{Number:} 1

\subparagraph*{Type:} Electrical complex signal

\subsection*{Examples}

\subsection*{Sugestions for future improvement}


